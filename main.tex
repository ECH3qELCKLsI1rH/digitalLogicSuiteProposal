\documentclass[a4paper,12pt]{article}

% --- Packages ---
\usepackage[utf8]{inputenc}
\usepackage{geometry}
\usepackage{graphicx}
\usepackage{hyperref}
\usepackage{setspace}
\usepackage{fancyhdr}
\usepackage{mathptmx} % Times New Roman font
\usepackage{titlesec}
\usepackage{array}
\usepackage{caption}

% --- Geometry: Margins ---
\geometry{
    a4paper,
    left=1.5in,
    right=1.25in,
    top=1in,
    bottom=1in,
    headheight=0.5in,
    headsep=0.5in,
    footskip=0.5in
}

% --- Line Spacing ---
\setstretch{1.5}

% --- Header & Footer ---
\pagestyle{fancy}
\fancyhf{}
\renewcommand{\headrulewidth}{0pt}
\renewcommand{\footrulewidth}{0pt}
\fancyhead[L]{Digital Logic Suite Proposal}
\fancyfoot[C]{\thepage}

% --- Section Title Formatting ---
\titleformat{\section}
  {\normalfont\bfseries\fontsize{18}{22}\selectfont}{\thesection}{1em}{}
\titleformat{\subsection}
  {\normalfont\bfseries\fontsize{16}{19}\selectfont}{\thesubsection}{1em}{}
\titleformat{\subsubsection}
  {\normalfont\bfseries\fontsize{14}{17}\selectfont}{\thesubsubsection}{1em}{}
\titleformat{\paragraph}[runin]
  {\normalfont\bfseries\fontsize{13}{15}\selectfont}{\theparagraph}{1em}{}

% --- Hyperref Setup ---
\hypersetup{
    colorlinks=true,
    linkcolor=blue,
    urlcolor=blue,
    pdftitle={Digital Logic Suite Proposal},
    pdfauthor={},
    pdfsubject={Project Proposal}
}

% --- Document Start ---
\begin{document}

% --- Title Page ---
\begin{center}
    \vspace*{2cm}
    {\LARGE \textbf{Digital Logic Suite Proposal}}\\[2cm]
    \textbf{Academic Project Proposal}\\[1cm]
    \vfill
    \textbf{Department of Electronics and Communication Engineering}\\
    \textbf{IOE, Pulchowk Campus}\\
    \textbf{July 1, 2025}\\
    \vspace*{2cm}
\end{center}
\thispagestyle{empty}
\newpage

% --- Acknowledgment ---
\section*{Acknowledgment}
We would like to express our sincere gratitude to our project supervisor, faculty members, and peers for their invaluable guidance and support throughout the development of the Digital Logic Suite project. Their encouragement and constructive feedback have been instrumental in shaping the direction and scope of this proposal.

\newpage

% --- Table of Contents ---
\tableofcontents
\newpage

% --- Introduction ---
\section{Introduction}
Digital logic design is fundamental to computer engineering and electronics. The Digital Logic Suite aims to provide a comprehensive, command-line-based toolkit for parsing and evaluating Boolean expressions, generating truth tables, minimizing logic using Karnaugh maps, simulating digital circuits, and visualizing logic structures. The suite leverages C++ object-oriented programming concepts and integrates with Graphviz for graphical representation.

% --- Objectives ---
\section{Objectives}
\begin{itemize}
    \item Develop a robust CLI tool for digital logic analysis and simulation.
    \item Implement Boolean expression parsing and evaluation.
    \item Generate truth tables for arbitrary Boolean expressions.
    \item Perform Karnaugh map minimization for logic simplification.
    \item Simulate digital circuits based on user-defined logic.
    \item Visualize logic circuits and truth tables using Graphviz.
    \item Demonstrate effective use of C++ OOP principles.
\end{itemize}

% --- Existing System (if any) ---
\section{Existing System}
Several tools exist for digital logic simulation and analysis, such as Logisim and online Boolean calculators. However, most are GUI-based and lack extensibility or CLI integration. Few open-source solutions provide a unified command-line interface with advanced features like expression parsing, minimization, and circuit visualization tailored for educational and research purposes.

% --- Proposed System ---
\section{Proposed System}
The proposed Digital Logic Suite is a modular, extensible CLI application written in C++. It supports parsing Boolean expressions, generating truth tables, Karnaugh map minimization, circuit simulation, and Graphviz-based visualization. The system is designed for ease of use, extensibility, and integration into academic workflows.

% --- Description ---
\section{Description}
The suite consists of several core modules:
\begin{itemize}
    \item \textbf{CLI Module:} Handles user input, command parsing, and help documentation.
    \item \textbf{Expression Parser:} Parses and evaluates Boolean expressions using recursive descent or shunting yard algorithms.
    \item \textbf{Truth Table Generator:} Produces truth tables for given expressions.
    \item \textbf{Karnaugh Map Minimizer:} Simplifies Boolean expressions using K-map techniques.
    \item \textbf{Circuit Simulator:} Simulates logic circuits based on parsed expressions.
    \item \textbf{Graphviz Visualizer:} Generates .dot files and invokes Graphviz to visualize logic circuits and truth tables.
\end{itemize}
The suite is implemented using C++ OOP concepts, with each module encapsulated as a class or set of classes.

% --- System Block Diagram ---
\section{System Block Diagram}
\begin{figure}[ht!]
    \centering
    % Placeholder for system block diagram
    % \includegraphics[width=0.9\textwidth]{images/system_block_diagram.png}
    \fbox{\parbox{0.9\textwidth}{\centering System Block Diagram Placeholder}}
    \caption{System Block Diagram: CLI $\rightarrow$ Expression Parser $\rightarrow$ Truth Table Generator $\rightarrow$ Circuit Simulator $\rightarrow$ Graphviz Visualizer}
    % Replace the placeholder image with the actual diagram as the project progresses.
\end{figure}

% --- Methodology ---
\section{Methodology}
The project will follow an iterative, modular development approach:
\begin{enumerate}
    \item Requirements analysis and system design.
    \item Implementation of core modules (expression parser, truth table generator).
    \item Integration of minimization and simulation features.
    \item Development of CLI and user documentation.
    \item Testing, debugging, and optimization.
    \item Integration with Graphviz for visualization.
\end{enumerate}

% --- Project Scope ---
\section{Project Scope}
The Digital Logic Suite is intended for academic and educational use. It focuses on combinational logic analysis, minimization, and simulation. Sequential logic and hardware synthesis are outside the current scope but may be considered for future extensions.

% --- Project Schedule ---
\section{Project Schedule}
\begin{table}[h!]
    \centering
    \begin{tabular}{|>{\raggedright}p{2cm}|p{10cm}|}
        \hline
        \textbf{Week} & \textbf{Planned Activities}                                   \\
        \hline
        Week 1        & Requirements gathering, literature review, and system design. \\
        \hline
        Week 2        & Implementation of CLI and Boolean expression parser.          \\
        \hline
        Week 3        & Truth table generation and Karnaugh map minimization modules. \\
        \hline
        Week 4        & Circuit simulation and Graphviz visualization integration.    \\
        \hline
        Week 5        & Testing, documentation, and final report preparation.         \\
        \hline
    \end{tabular}
    \caption{Project Schedule for Digital Logic Suite}
\end{table}

% --- Libraries and Tools ---
\section{Libraries and Tools}
\begin{itemize}
    \item \textbf{C++ Standard Template Library (STL):} Data structures and algorithms.
    \item \textbf{Graphviz:} Visualization of logic circuits and truth tables.
    \item \textbf{Visual Studio:} Development environment.
    \item \textbf{Git:} Version control and collaboration.
\end{itemize}

% --- Header Files List ---
\section{Header Files List}
The following C++ header files will be used throughout the project:
\begin{verbatim}
#include <iostream>
#include <string>
#include <vector>
#include <map>
#include <algorithm>
#include <stack>
#include <queue>
#include <set>
#include <fstream>
#include <sstream>
\end{verbatim}

% --- Placeholders for Appendices, Class Diagrams, Sample Outputs ---
\section*{Appendices}
% Add appendices here as needed.

\section*{Class Diagrams}
% Insert class diagrams here (use \includegraphics if available).

\section*{Sample Outputs}
% Add sample outputs or screenshots here.

% --- References ---
\bibliographystyle{plain}
\bibliography{references}

\end{document}
