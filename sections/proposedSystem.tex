\section{Proposed System}

The proposed Digital Logic Suite will be a command-line interface (CLI) application built using the C++ programming language. It will be organized into separate modules, each responsible for a specific task related to digital logic design. By using object-oriented programming (OOP), we will keep the system modular, making it easy to maintain and extend with new features in the future.

The key features of the proposed system are:

\begin{itemize}
    \item \textbf{CLI Module:} This module will handle all user inputs, command parsing, and display of help messages. It will provide a simple and user-friendly interface where users can enter commands, view error messages, and access documentation directly from the terminal.

    \item \textbf{Boolean Expression Parser:} This module will parse and evaluate Boolean expressions entered by the user. It will support different operators such as AND, OR, NOT, NAND, NOR, XOR, and XNOR, along with parentheses for grouping expressions. The parser will ensure correct syntax and operator precedence, providing immediate feedback on invalid expressions.

    \item \textbf{Truth Table Generator:} Once a Boolean expression is parsed, this module will generate the corresponding truth table. It will show all possible input combinations and the resulting output, helping users understand the behavior of their logic circuits.

    \item \textbf{Karnaugh Map Minimizer:} This module will simplify Boolean expressions using Karnaugh map (K-map) techniques. It will display each step of the minimization process, including grouping of 1s or 0s, and produce the simplified expression. This will help users learn how K-map simplification works.

    \item \textbf{Circuit Simulator:} Based on the parsed or minimized expressions, this module will simulate the behavior of the corresponding digital logic circuit. It will accept user-defined inputs and instantly display the outputs, allowing users to verify the correctness of their designs without building physical circuits.

    \item \textbf{Graphviz Visualizer:} To make it easier to see and understand logic circuits, this module will create visual diagrams of circuits and truth tables using Graphviz. It will generate .dot files that can be converted into images or PDFs showing the logic gates and connections.

    \item \textbf{Modular Design:} Each module will be implemented as a separate class or set of classes. This design makes the system modular and extensible, so future features—like sequential circuit simulation or support for memory elements—can be added easily without changing the existing modules.
\end{itemize}

Together, these modules will provide a complete solution for digital logic analysis and simulation from the command line. The suite will enable users to analyze logic, simplify expressions, generate truth tables, simulate circuits, and visualize designs all in one tool, without requiring any graphical interface.

The next sections will describe the system in detail, including a block diagram, the methodology we will follow, the scope of the project, the schedule, and the libraries and tools used.

\clearpage
\subsection{Description}

The Digital Logic Suite will consist of several key modules, each designed to handle a specific task in the analysis and simulation of digital logic. Below is a detailed description of each module and its purpose:

\begin{itemize}
    \item \textbf{Command-Line Interface (CLI) Module:} This module will be responsible for taking user input and interpreting commands. It will include features like command history, auto-suggestions, detailed help documentation, and clear error messages. The goal is to make it easy for users to interact with the suite even if they are new to command-line tools.

    \item \textbf{Expression Parser:} This module will analyze Boolean expressions entered by the user. It will check for correct syntax, evaluate the expression, and convert it into an internal data structure that can be used by other modules. The parser will support common operators like AND, OR, NOT, as well as more advanced ones such as NAND, NOR, XOR, and XNOR.

    \item \textbf{Truth Table Generator:} Once a Boolean expression is parsed, this module will calculate and display the complete truth table. It will show all possible input combinations for the variables involved and the corresponding output, giving users a clear view of the logic they have defined.

    \item \textbf{Karnaugh Map Minimizer:} This module will take the generated truth table or directly use the parsed Boolean expression to create a Karnaugh map (K-map). It will automatically group cells in the K-map to simplify the expression and display the step-by-step process so users can follow how the minimization was done.

    \item \textbf{Circuit Simulator:} Based on the original or minimized expression, this module will simulate the digital logic circuit. It will allow users to enter specific inputs and see the resulting outputs in real-time, helping them confirm the correctness of their logical designs without needing hardware.

    \item \textbf{Graphviz Visualizer:} This module will convert the parsed or minimized Boolean expression into a visual representation using Graphviz. It will generate .dot files that can produce diagrams showing the gates and connections involved in the logic circuit, making it easier for users to understand complex logic visually.

    \item \textbf{Data Export and Reporting:} The suite will include options to export generated truth tables, minimized expressions, and visual diagrams to common formats like CSV, TXT, and PDF. This feature will allow students and professionals to include the outputs in their reports, assignments, or documentation.
\end{itemize}

All these modules will work together in a modular architecture, where each module can be developed, tested, and improved separately. By keeping the modules independent, it will be easy to add new features or make updates without affecting other parts of the suite.

\clearpage

\subsection{System Block Diagram}
\begin{tikzpicture}[
  node distance=1.8cm and 3.5cm,
  box/.style={
    draw,
    rounded corners,
    minimum width=3.2cm,
    minimum height=1.2cm,
    align=center,
    fill=blue!5,
    thick
  },
  ->, >=Stealth
]

% First row
\node[box] (cli) {Command \\ Line \\ Interface};
\node[box, right=of cli] (parser) {Expression \\ Parser};

% Second row
\node[box, below=of parser] (ttg) {Truth Table \\ Generator};
\node[box, left=of ttg] (kmap) {K-map \\ Minimizer};

% Third row
\node[box, below=of kmap] (sim) {Circuit \\ Simulator};
\node[box, right=of sim] (viz) {Graphviz \\ Visualize};

% Fourth row: center below sim and viz
\node[box] (cd) at ($(sim)!0.5!(viz) - (0,2.2)$) {Generated \\ Circuit \\ Diagram};

% Arrows with labels
\draw (cli) -- node[midway, above] {\small User Input} (parser);
\draw (parser) -- node[midway, right] {\small Parsed Expression} (ttg);
\draw (ttg) -- node[midway, above] {\small Truth Table Generator} (kmap);
\draw (kmap) -- node[midway, left] {\small Minimized Expression} (sim);
\draw (sim) -- node[midway, above] {\small Simulated Circuit} (viz);

% Arrows to centered last node
% \draw (viz) -- ($(cd.north east)+(-0.2,0)$);
\draw (viz) |- ($(cd.north east)+(-0.2,0)$);

\end{tikzpicture}

The overall architecture of the Digital Logic Suite can be represented by a block diagram showing how user inputs flow through different modules to produce outputs like truth tables, minimized expressions, simulated results, and visual diagrams.


\noindent The block diagram shows the following flow:
\begin{itemize}
    \item Users interact with the \textbf{Command-Line Interface (CLI)} to enter Boolean expressions or commands.
    \item The \textbf{Expression Parser} processes these expressions, checks syntax, and creates internal data structures.
    \item The \textbf{Truth Table Generator} uses the parsed expression to produce a complete truth table showing all input-output possibilities.
    \item The \textbf{Karnaugh Map Minimizer} takes the truth table and simplifies the logic using K-map techniques, generating minimized expressions.
    \item The \textbf{Circuit Simulator} uses the original or minimized expressions to simulate circuit behavior and outputs results.
    \item The \textbf{Graphviz Visualizer} generates visual diagrams of logic circuits and truth tables based on data from the parser, truth table, or minimizer.
\end{itemize}

This modular design ensures each component can operate independently while still contributing to the overall system, making the suite easy to develop, maintain, and extend.
